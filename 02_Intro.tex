\chapter{Introduction to Si3D-L}
\section{Scientific basis, main features}




\section{List of publications Si3D-L}

The numerical model has been applied to multiple lakes and reservoirs around the world. A list of publications available online is shown next:  

\begin{enumerate}
    \item Rueda, Francisco J. (2001). A Three-Dimensional Hydrodynamic and Transport Model for Lake Environments. University of California Davis.
    \item Rueda, F. J., \& Schladow, S. G. (2002). Quantitative Comparison of Models for Barotropic Response of Homogeneous Basins. Journal of Hydraulic Engineering, 128(2), 201–213.
    \item Rueda, F. J., \& Schladow, S. G. (2002). Surface seiches in lakes of complex geometry. Limnology and Oceanography, 47(3), 906–910.
    \item Rueda, F. J., Schladow, S. G., Monismith, S. G., \& Stacey, M. T. (2003). Dynamics of large polymictic lake. I: Field observations. Journal of Hydraulic Engineering, 129(2), 82–91.
    \item Rueda, F. J., Schladow, S. G., \& Pálmarsson, S. Ó. (2003). Basin-scale internal wave dynamics during a winter cooling period in a large lake. Journal of Geophysical Research C: Oceans, 108(3), 42–1.
    \item Rueda, F. J., Schladow, S. G., Monismith, S. G., \& Stacey, M. T. (2005). On the effects of topography on wind and the generation of currents in a large multi-basin lake. Hydrobiologia, 532(1), 139–151.
    \item Rueda, F. J., \& Cowen, E. A. (2005). Residence time of a freshwater embayment connected to a large lake. Limnology and Oceanography, 50(5), 1638–1653.
    \item Smith, Peter. (2006). A Semi-Implicit, Three-Dimensional Model for Estuarine Circulation. USGS Report 176.
    \item Schladow, S. G., Rueda, F. J., Fleenor, E., \& Chung, E. G. (2006). Predicting the effects of configuration changes on Salton Sea stratification using a three-dimensional hydrodynamic model. TERC Rep. 06-007, Report to CH2M HILL and California Department of Water Resources.
    \item Rueda, F. J., Fleenor, W. E., \& de Vicente, I. (2007). Pathways of river nutrients towards the euphotic zone in a deep-reservoir of small size: Uncertainty analysis. Ecological Modelling, 202(3–4), 345–361.
    \item Rueda, Francisco J., S. Geoffrey Schladow, \& Jordan F. Clark. (2008). “Mechanisms of Contaminant Transport in a Multi-Basin Lake.” Ecological Applications: A Publication of the Ecological Society of America 18(8 Suppl): A72-88.
    \item Rueda, F. J., \& MacIntyre, S. (2009). Flow paths and spatial heterogeneity of stream inflows in a small multibasin lake. Limnology and Oceanography, 54(6), 2041–2057.
    \item Rueda, F. J., Vidal, J., \& Schladow, G. (2009). Modeling the effect of size reduction on the stratification of a large wind-driven lake using an uncertainty-based approach. Water Resources Research, 45(3), 1–15.
    \item Llebot, Clara. (2010). “Interactions between Physical Forcing, Water Circulation and Phytoplankton Dynamics in a Microtidal Estuary.” 212.
    \item Doyle, L. (2010). “A Three-Dimensional Water Quality Model for Estuary Environments.” 213.
    \item Rueda, F. J., \& MacIntyre, S. (2010). Modelling the fate and transport of negatively buoyant storm-river water in small multi-basin lakes. Environmental Modelling and Software, 25(1), 146–157.
    \item Ramón, Cintia L., Joan. Armengol, Josep. Dolz, Jordi. Prats, and Francisco J. Rueda. (2014). Mixing Dynamics at the Confluence of Two Large Rivers Undergoing Weak Density Variations. Journal of Geophysical Research: Oceans 119(4):2386–2402.
    \item Hoyer, Andrea B., Marion E. Wittmann, Sudeep Chandra, S. Geoffrey Schladow, \& Francisco J. Rueda. (2014). “A 3D Individual-Based Aquatic Transport Model for the Assessment of the Potential Dispersal of Planktonic Larvae of an Invasive Bivalve.” Journal of Environmental Management 145:330–40.
    \item Acosta, M., Anguita, M., Fernández-Baldomero, F. J., Ramón, C. L., Schladow, S. G., \& Rueda, F. J. (2015). Evaluation of a nested-grid implementation for 3D finite-difference semi-implicit hydrodynamic models. Environmental Modelling \& Software, 64, 241–262.
    \item Ramón, C. L., A. Cortés, \& Francisco J. Rueda. (2015). “Inflow–Outflow Boundary Conditions along Arbitrary Directions in Cartesian Lake Models.” Computers \& Geosciences 74:87–96.
    \item Hoyer, A. B., Schladow, S. G., \& Rueda, F. J. (2015). A hydrodynamics-based approach to evaluating the risk of waterborne pathogens entering drinking water intakes in a large, stratified lake. Water Research, 83, 227–236.
    \item Hoyer, A. B., Schladow, S. G., \& Rueda, F. J. (2015). Local dispersion of nonmotile invasive bivalve species by wind-driven lake currents. Limnology and Oceanography, 60(2), 446–462.
    \item Ramón, Cintia L., Jordi Prats, \& Francisco J. Rueda. (2016). The Influence of Flow Inertia, Buoyancy, Wind, and Flow Unsteadiness on Mixing at the Asymmetrical Confluence of Two Large Rivers. Journal of Hydrology 539:11–26.
    \item Chen, Shengyang, Chengwang Lei, Cayelan C. Carey, Paul A. Gantzer, \& John C. Little. (2017). “A Coupled Three-Dimensional Hydrodynamic Model for Predicting Hypolimnetic Oxygenation and Epilimnetic Mixing in a Shallow Eutrophic Reservoir.” Water Resources Research 53(1):470–84.
    \item Chen, Shengyang, John C. Little, Cayelan C. Carey, Ryan P. McClure, Mary E. Lofton, \& Chengwang Lei. (2018). “Three‐Dimensional Effects of Artificial Mixing in a Shallow Drinking‐Water Reservoir.” Water Resources Research 54(1):425–41.
    \item Valbuena, S. A., Bombardelli, F. A., Cort\'{e}s, A., Largier, J. L., Roberts, D. C., Forrest, A. L., \& Schladow, S. G. (2022). 3D Flow Structures During Upwelling Events in Lakes of Moderate Size. Water Resources Research, 58(3), 1–35.
\end{enumerate}


\section{Manual summary} 