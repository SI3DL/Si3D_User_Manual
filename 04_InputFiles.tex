\section{Description of Input Files}

The purpose of this section is to provide guidance on the input files required to use the Si3D-L model. To properly use Si3D-L, the input files must reside within the same directory as the executable file of the model (i.e., 'psi3d'). This directory is where output files will be created. The simplest model requires the input file ('si3d\_inp.txt'), the bathymetry file ('h'), and the initial condition file ('si3d\_init.txt') to properly run the numerical model Si3D-L. This simplest model considers only an enclosed basing forced by wind stress and without a heat source at the surface boundary. However, the required input files for the numerical model to run properly variate based on the application. For instance, if heating/cooling from the atmosphere is modeled, the 'surfbc.txt' file is required (see \ref{S:surfbc}). In addition, if open boundaries are used, such as river inflows, the corresponding 'openbc\#\#.txt' file is needed for each open boundary in the model (see \ref{S:openbc}).

\subsection{Main Input File}
\label{S:Main}

The main input file in Si3D-L defines model-controlling parameters that determine the model characteristics, numerical techniques for solving the governing equations, and output types for post-processing purposes. The input file name must be named "si3d\_inp.txt", and an example of the file is provided within the "\href{https://github.com/SI3DL/psi3d/tree/main/SampleFiles}{SampleFiles}" folder named "si3d\_inp\_master.txt". 

The input file has a total of 12 sections to edit. In all these sections, proper formatting of each line within the file is REQUIRED for proper use of the Si3D-L suite. Otherwise, Si3D-L will run into a reading error, and the model will stop. Each user-editable line within the file follows:

\begin{table}[h]
    \centering
    \begin{tabular}{c c c}
        Variable\_Name & ! Variable\_Value & ! Variable\_Description
    \end{tabular}
    \label{tab:inputformat}
\end{table}

\noindent where Variable\_Name must have a length of 14 spaces (including the ! sign) and defines the variables being edited, Variable\_Value must have 20 spaces and assigns the values to the corresponding variable name, and Variable\_Description provides a brief explanation of the variable's meaning. In the following subsections, an explanation of each variable within the input file is presented along with the different options that Si3D-L suite offers.

\subsubsection{Model Title}
Line 2 of the file is available for users to edit with the model run name of their application. 
\subsubsection{Start date and time for simulations}
This section is provided to the user to define the starting date of the model run. The section considers the definition of the year, month, day, and hour. For the hour, the format must follow military time (e.g., 0300 for 3:00 am, 1800 for 6:00 pm). 
\subsubsection{Space-time domains, cell size \& time step}
This section is provided for the user to define the space and time domains of the numerical model. The variables to be defined are:

\begin{itemize}
    \item $xl\ -\ [m]$ stands for the length of the model domain in the $EW$ direction or $x$ coordinate. This variable divided by $idx$ must be equal to the $imx$ parameter within the bathymetry file.
    \item $yl\ -\ [m]$ stands for the length of the model domain in the $SN$ direction or $y$ coordinate. This variable divided by $idy$ must be equal to the $jmx$ parameter within the bathymetry file.
    \item $zl\ -\ [m]$ stands for the maximum depth of the model or the length of the domain in the vertical direction $z$.
    \item $tl\ -\ [sec]$ stands for the total time of model run. This should be at least equal to the number of hours defined in the input files for open boundaries and surface boundary conditions.
    \item $idx\ -\ [m]$ stands for the cell size of the model domain in the $x$ coordinate.
    \item $idy\ -\ [m]$ stands for the cell size of the model domain in the $y$ coordinate.
    \item $idz\ -\ [m]$ stands for the cell size of the model domain in the $z$ coordinate. This value is used within the numerical model when a constant layer thickness is used to define the vertical mesh size. 
    \item $dzmin\ -\ [m]$ stands for the minimum cell size in the vertical direction. This value is used when .... \textcolor{red}{NEEDS TO BE COMPLETED}
    \item $datadj\ -\ [m]$ variable to adjust the datum of the depths within the bathymetry file. The default value is $0.0$
    \item $zeta0\ -\ [m]$ variable to define the initial location of the water surface with respect to $z = 0$
    \item $idt\ -\ [sec]$ stands for the time step used to solve the numerical model. The user must consider the Courant–Friedrichs–Lewy criterion to define this parameter based on the model application.
    \item $ibathf\ -\ [-]$ defines if constant or variable layer thickness in the vertical mesh size is used. $ibathf = 0$ for constant layer thickness, and $ibathf = -1$ for variable layer thickness in the vertical direction. If $ifbathf = -1$, a 'si3d\_layer.txt' file, detailing the thicknesses of each layer, is needed as an input file (see \ref{S:layer})
\end{itemize}

\subsubsection{Parameters controlling solution algorithm}
This section allows users to define the parameters used within the numerical model to solve the governing equations. The variables to be defined are: 

\begin{itemize}
    \item $itrap\ -$ defines whether the trapezoidal iteration or the single-step Leap Frog method is used in the solution of the finite difference discretization. $itrap = 0$ for single-step Leap Frog, and $itrap = 1$ for trapezoidal iterations.
    \item $niter\ -$ defines the number of trapezoidal iterations to be used in the numerical methods. The greater the number of iterations, the longer time for the numerical model to find a solution for a time step. 
    \item $smooth\ -$ binary parameter for smoothing of Leap Frog solution. $1$ is on and $0$ is no smoothing.
    \item $beta\ -$ parameter controlling the smoothing filter of the solution. $beta = 0.05-0.2$ is recommended.
    \item $iturb\ -$ parameter controlling the method used for the turbulent closure method within the governing equations for the vertical direction. $iturb = 0$ for constant vertical eddy viscosity and diffusivity, and $iturb = 1$ to use the implemented 2-equation Mellor \& Yamada turbulent closure method.
    \item $az0\ -\ [m2/s]$ constant vertical eddy viscosity. Used only when $iturb = 0$
    \item $dz0\ -\ [m2/s]$ constant vertical eddy diffusivity. Used only when $iturb = 0$
    \item $iadv\ -$ binary parameter controlling if advection terms within the momentum governing equation are solved. $iadv = 0 \rightarrow$ OFF, and $iadv = 1 \rightarrow$ ON.
    \item $itrmom\ -$ parameter controlling the algorithm used for solving the momentum horizontal advection. $itrmom = 1 \rightarrow$ centered scheme, $itrmom = 2 \rightarrow$ upwind scheme, $itrmom = 3 \rightarrow $, and $itrmom = 4 \rightarrow $ ... \textcolor{red}{TO BE COMPLETED}
    \item $ihd\ -$ parameter controlling horizontal diffusive terms within the governing equations. $ihd = 0 \rightarrow$ OFF, $ihd = 1 \rightarrow$ constant values, and $ihd = 2 \rightarrow $ Smagorinsky closure scheme.
    \item $ax0\ -\ [m2/s]$ constant horizontal eddy diffusivity and viscosity in the $EW$ direction or $x$ coordinate. This value is used when $ihd = 1$ for the governing equations of the velocity field, and at all times for the solution of the scalar transport equation.
    \item $ay0\ -\ [m2/s]$ constant horizontal eddy diffusivity and viscosity in the $SN$ direction or $y$ coordinate. This value is used when $ihd = 1$ for the governing equations of the velocity field, and at all times for the solution of the scalar transport equation.
    \item $f \ -\ [1/s]$ constant value that defined the Coriolis frequency parameter.
    \item $ theta\ -$ defines the weighting parameter for the semi-implicit solution. \textcolor{red}{TO BE COMPLETED}
    \item $ibc\ -$ binary controlling parameter defining if baroclinc terms are included in the solution of the governing equation. $ibc = 0 \rightarrow$ OFF, and $ibc = 1 \rightarrow$ ON
    \item $isal\ -$ binary parameter controlling if scalar transport equations is solved. $isal = 0 \rightarrow$ OFF, and $isal = 1 \rightarrow$ ON.
    \item $itrsch\ -$ parameter controlling the algorithm used for solving the scalar transport equation. $itrsch = 1 \rightarrow$ centered scheme, $itrsch = 2 \rightarrow$ upwind scheme, $itrsch = 3 \rightarrow$ $u$ at layer $k = k1z + 1$, and $itrsch = 4 \rightarrow$ for flux limiter scheme. \textcolor{red}{NEEDS TO BE VERIFIED, I BELIEVE ONLY THE FLUX LIMITED IS USED}
    \item $cd\ -$ parameter that defines the bottom drag coefficient of the model bottom. This value is constant in the whole domain.
    \item $ifsbc\ -$ parameter controlling the type of surface boundary condition to be used in the solution of the numerical model. $ifsbc = 0 \rightarrow$ constant wind velocity w/o heat source. $ifsbc = 1 \rightarrow$ pre-process mode (i.e., heat budget prior to model run. $ifsbc = 2 \rightarrow$ runtime mode I, cloud cover as input. $ifsbc = 3 \rightarrow$ runtime II, incoming longwave is used as input. $ifsbc = 10 \rightarrow$ spatially variable runtime I. $ifsbc = 11 \rightarrow$ spatially variable runtime II. $ifsbc = 20 \rightarrow$ variable wind speed w/o heat sources. We refer the reader to Section \ref{S:surfbc} for more details on the different type of surface boundary conditions.
    \item $dtsbc\ -\ [sec]$ parameter controlling the time step of the records used within the surface boundary conditions file (i.e., 'si3d\_surbc.txt'). Only used if $ifsbc > 0$
    \item $cw\ -$ defines the wind drag coefficient for the quadratic stress law implemented for the wind-induced shear stress. This parameter is only used when $ifsbc = 0$
    \item $ws\ -\ [m/s]$ defines the wind speed for the numerical model when $ifsbc = 0$
    \item $phi\ -\ [^{\circ}]$ defines the wind direction in degrees where the wind is coming from. This parameter is only used when $ifsbc = 0$
    \item $idbg\ -$ binary parameter controlling if log messages are within the output of the model. This parameter is used for debugging purposes. $idbg = 0 \rightarrow$ OFF, and $idbg = 1 \rightarrow$ ON.
    \item $nth\ -$ parameter controlling the number of threads used for the parallelization of the model. This number coincides with the number of subdomains into which the model domain is divided into.
\end{itemize}

\subsubsection{Output specifications for time files}

\subsubsection{Output specifications for horizontal planes}

\subsubsection{Output specifications for 3D domain files}

\subsubsection{Open boundary condition specifications}

\subsubsection{Open boundary conditions for nesting procedures}

\subsubsection{Specification for water quality and tracer simulation}

\subsubsection{Specification for oxygen system simulations}

\subsubsection{Specification for interpolation method}


\subsection{Surface Boundary Condition File}
\label{S:surfbc}


\subsection{Open Boundary Condition File}
\label{S:openbc}

\subsection{Layer file}
\label{S:layer}